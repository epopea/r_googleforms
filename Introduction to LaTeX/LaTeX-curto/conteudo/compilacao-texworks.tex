% !TeX root = ../latex-curto.tex
% !TeX encoding = utf8

\section{Pondo a mão na massa}

\begin{frame}

\frametitle{Editor padrão: \TeX works}

\imagem[height=6.5cm,keepaspectratio]{texworks-janela.png}

\end{frame}

\begin{frame}

\frametitle{Agora faça você}

Abra o programa TeXworks e digite\medskip

  \framebox{\begin{code*}\scriptsize
    % \documentclass[12pt]{article} % preâmbulo
%
    % \usepackage[utf8]{inputenc} % uso de acentuação
    % \usepackage[brazil]{babel} % hifenização
%
%
    % \begin{document} % corpo do texto
%
    % Oi.  Este é meu 1º documento em \LaTeX.
%
    % \end{document}
    \purple{\string\documentclass}\green{[12pt]}\blue{\ac{}article\fc{}}\ \ \ \ \ \gray{\textit{\pc{}\ preâmbulo}}\\
    \\
    \purple{\string\usepackage}\green{[utf8]}\blue{\ac{}inputenc\fc{}}\
    \ \ \ \ \ \ \gray{\textit{\pc{}\ uso\ de\ acentuação}}\\
    \purple{\string\usepackage}\green{[brazil]}\blue{\ac{}babel\fc{}}\ \ \ \ \ \ \ \ \gray{\textit{\pc{}\ hifenização}}\\
    \\
    \purple{\string\begin}\blue{\ac{}document\fc{}}\ \ \ \ \ \ \ \ \ \ \ \ \ \ \ \ \ \ \gray{\textit{\pc{}\ corpo\ do\ texto}}\\
      \\
      Oi.\ \ Este\ é\ meu\ 1º\ documento\ em\ \purple{\string\LaTeX}.\\
      Calcular\ o\ volume\ dum\ paralelepípedo\ é\ trivial.\\
      \\
      \purple{\string\end}\blue{\ac{}document\fc{}}
  \end{code*}}


\bigskip
Crie uma pasta\\
 e salve este arquivo nela como \green{\texttt{primeiro.tex}}.

\end{frame}

\begin{frame}
\frametitle{Rodando o \LaTeX}

  O processo é feito no TeXworks.

  \begin{itemize}
  \item Salve o arquivo \texttt{.tex}
  \item Para ``rodar o \LaTeX'', clique no botão
    \imagem[height=12pt,keepaspectratio]{botao-latex.png}\\ \imagem[height=40pt,keepaspectratio]{texworks-painel.png}
  \item Se não houveram erros, parabéns!!
  \item O visualizador PDF integrado aparecerá.

  \end{itemize}

\end{frame}

\begin{frame}
\frametitle{Compilação SEM erros}
Se compilou bem, \purple{a janela de compilaçao desaparece no final.}

\begin{center}
\imagem[width=4.5cm,keepaspectratio]{texworks-compilando.png}\ \ \ \ 
\imagem[width=4.5cm,keepaspectratio]{texworks-rodoubem.png}
\end{center}
\end{frame}

\begin{frame}
\frametitle{Compilação COM erros}
No final, a janela fica, falando a linha (aproximada) do erro.
\begin{center}
\imagem[width=8cm,keepaspectratio]{texworks-erro.png}
\end{center}
\end{frame}

\section[TeXworks]{\TeX works}

\begin{frame}
  \frametitle[Comentários mágicos no TeXworks]{Comentários mágicos no
    \TeX works}
  

  \begin{dica}{Acrescente as linhas no topo dos arquivos \texttt{.tex}}
    \begin{itemize}
    \item \purple{\texttt{\% !TEX encoding = utf8}} \\
      força o \TeX works a abrir com codificação
      certa\footnote{... no PC do seu orientador \smiley}
    \item \purple{\texttt{\% !TEX root = \textit{arquivo}}}\\
      declara arquivo raiz;\\
      compilação funciona desde qualquer arquivo
    \end{itemize}
  \end{dica}
\end{frame}

\begin{frame}
  \frametitle{Mais dicas no \TeX works}

  \begin{description}
  \item[Realce de sintaxe] Menu Formato $\to$ Realce de sintaxe $\to$
    $\bullet$~LaTeX.
  \item[aspas] Menu Formato $\to$ Aspas automáticas $\to$ $\bullet$~Unicode characters.
  \item[Preferências] Altere também estas preferências no menu Editar
    $\to$ Preferências (reinicie o editor).
  \end{description}

\end{frame}

%%% Local Variables: 
%%% mode: latex
%%% TeX-master: "../latex-curto"
%%% End: 
