% !TeX root = ../latex-curto.tex
% !TeX encoding = utf8

\section{Comandos}


\begin{frame}
  \frametitle{Comandos}

  \begin{block}{Definindo comandos}
    %\centering
    \texttt{\mbox{\ \ }\blue{\string\newcommand}\ac{}\red{\textit{\string\comando}}\fc{}\purple{[\textit{nº args}]}\ac{}\textit{substituição}\fc{}}\\

    \begin{itemize}
    \item \texttt{\#1} --- primeiro parâmetro
    \item \texttt{\#2} --- segundo parâmetro
    \item \dots
    \end{itemize}

  \end{block}
\end{frame}

\begin{frame}
  \frametitle{Exemplos}

  \begin{exemplo}[comando sem argumento]
    \medskip\par
    \texttt{\blue{\string\newcommand}\ac{}\green{\string\R}\fc{}\ac{}\purple{\string\mathbb\ac{}R\fc{}}\fc{}}\medskip

  % Define o comando sem parâmetros\smallskip

  %   \centerline{\texttt{\string\R} $\to$ \texttt{\string\mathbb\ac{}R\fc{}}}

    \bigskip

    \texttt{Seja\ \dolar{}a\string\in\green{\string\R}\dolar{}\ tal\ que\ ...}\smallskip

    \newcommand{\R}{\mathbb{R}}
    Seja $a\in\R$ tal que ...
  \end{exemplo}

\end{frame}


\newcommand{\V}[1]{(#1_1,#1_2)}

\begin{frame}
  \frametitle{Exemplos}

  \begin{exemplo}[comando com 1 argumento]
    Suponha que se use muitas vezes o par $(x_1,x_2)$, $(y_1,y_2)$,
    $(k_1,k_2)$ etc.
    \medskip 

    \texttt{\blue{\string\newcommand}\ac{}\green{\string\V}\fc\red{[1]}\ac{}{(\purple{\#1}\us{}1,\purple{\#1}\us{}2})\fc{}}\medskip

    
    \bigskip

    \texttt{...\ considere o vetor \dolar{}\green{\string\V}\purple{\ac\string\theta\fc}\dolar{}\
       ...}\smallskip

    
     \dots{} considere o vetor $\V{\theta}$ \dots
  \end{exemplo}
\end{frame}






%%% Local Variables: 
%%% mode: latex
%%% TeX-master: "../latex-curto"
%%% End: 
